\chapter{Introdução}
\label{chap:Introducao}

Os cursos de Engenharia elétrica trabalham com os estudos e aplicações da eletricidade, eletromagnetismo e eletrônica, sendo divididas em várias subáreas, como: Sistemas de energia elétrica ou sistemas de potência; Sistemas de eletrônica de potência; Sistemas de controle e automação; Sistemas de eletrônica e instrumentação; Sistemas de microeletrônica; Sistemas de telecomunicações; Sistemas biomédicos.

Segundo Macedo et al. \citeyearpar{macedo2012novas}, hoje em dia, a comunicação e a educação encontram-se interligadas no mundo digital.
Por isso, professores e alunos devem utilizar, adequadamente, os recursos dessas novas
tecnologias, explorando seu potencial pedagógico e utilizando, de forma positiva, esses novos
ambientes de ensino e aprendizagem.

A preocupação com a formação acadêmica dos discentes da disciplina Subestações de Energia Elétrica foi a proposta principal deste trabalho. As aulas ministradas nos semestres anteriores, utilizando softwares de simulação não didáticos, como o MGA Power Simulator, ou não apropriados para a disciplina, como o CADe SIMU, trouxe a motivação para a implementação de uma ferramenta pedagógica que atendesse as necessidades básicas desta disciplina, pelo fato das tecnologias nesta área serem escassas e/ou não atenderem as necessidades da disciplina.


%%=============================================================================
\section{Justificativa}

A proposta deste trabalho é utilizar um ambiente de desenvolvimento integrado (IDE -  Integrated Development Environment) para executar a construção de uma aplicação com uma interface gráfica do usuário (GUI – Graphic User Interface), capaz de proporcionar ao usuário demonstrações de manobras na planta, apresentando os diferentes equipamentos que compõem um subestação, e seus arranjos físicos.


%%=============================================================================
\section{Objetivos}

\subsection{Objetivo Geral}

O objetivo geral desse trabalho é o desenvolvimento de uma aplicação educacional que dê suporte à disciplina de Subestações de Energia Elétrica. Além disso, espera-se que esta facilite no processo de ensino-aprendizagem na UFSM, campus Cachoeira do Sul. 


\subsection{Objetivo Específico}

Os objetivos específicos desse trabalho são:

\begin{itemize}
\begin{itemize}
    \item Estudo elaborado sobre subestações, seus arranjos e suas manobras, bem como uma consulta com relação as exigências que o algoritmo deveria atender.
    
    \item Definir as ferramentas utilizadas para a construção do projeto.

    \item Definir a estrutura da interface com o utilizador
 
    \item  Testar a aplicação nas disciplinas de Subestações de Energia Elétrica e avaliar o interesse e aprendizado dos alunos o seu uso.
    
\end{itemize}
\end{itemize}
%%=============================================================================
\section{Estrutura do Trabalho}

Este trabalho está dividido em XXX capítulos. O presente capítulo destina-se a uma breve introdução da necessidade de desenvolvimento de uma aplicação pedagógica para a didática de subestações de energia elétrica, a justificativa e motivação para sua realização e os objetivos almejados.

O capítulo 2 é dedicado aos %distúrbios elétricos que podem ocorrer no SIN, especialmente o fenômeno de sobretensão. O conhecimento das influências destes transitórios no transformador de potência serve como ponto de partida para elaboração de projetos bem sucedidos. Ainda neste capítulo, são apresentados os tipos de enrolamentos de transformadores empregados para Alta Tensão (AT), bem como suas características ressonantes. Por fim, são abordados os cálculos e teoria para desenvolver uma modelagem de transformador adequada para altas frequências, destacando a importância da realização de design review.

O capítulo 3 é integralmente dedicado a discussão %da modelagem do equipamento, detalhando os parâmetros presentes e como obtê-los  através das características construtivas do transformador. Além disso, propõe o circuito equivalente e demonstra o método a ser utilizado para representação deste circuito no SPICE (Simulation Program with Integrated Circuit Emphasis) para simulação computacional.

O capítulo 4 apresenta, brevemente, o modelo %do transformador calculado através da fundamentação teórica apresentada nos capítulos anteriores. A escolha do transformador utilizado no estudo se deve a disponibilidade do conhecimento detalhado deste projeto, oriundo do design review.

No capítulo 5, abordam-se alguns estudos de caso %para o modelo desenvolvido, buscando evidenciar suas potencialidades para a simulação e análise de transitórios eletromagnéticos em transformadores.

...

Por fim, o capítulo xxx apresenta as conclusões deste trabalho e sugestões para o desenvolvimento de trabalhos futuros.

%%=============================================================================


%Conforme Sebesta \citeyearpar{Sebesta:2005}, uma boa linguagem de programação é Java \cite{Sun:2010}.

%\begin{quote}
         %Contexto é qualquer informação que pode ser utilizada para caracterizar a situação de uma entidade. Uma entidade é uma pessoa, lugar ou objeto que          podem ser considerados relevantes para a interação entre um usuário e uma aplicação, incluindo o usuário e as suas próprias aplicações. \citep[tradução nossa]{Abowd:1999}
%\end{quote}

%Outras referências: \cite{Alex:2010}, \cite{Weiser:1991} e \cite{norell:thesis}.


% Um exemplo de tabela é a \ref{tab:curry}:

% \begin{table}[!hbt] % [htb]-> here, top, bottom
   % \centering   % tabela centralizada
   % \setlength{\arrayrulewidth}{1\arrayrulewidth}  % espessura da  linha
   % \setlength{\belowcaptionskip}{5pt}  % espaço entre caption e tabela
   % \caption{Correspondência \textit{Curry-Howard}}
   % \begin{tabular}{l|l} % c=center, l=left, r=right 
      % \hline
      % \textbf{Lógica} & \textbf{Linguagens de Programação} \\
      % \hline
      % proposições & tipos  \\
      % proposição $P \supset Q$ & tipo $P \rightarrow Q$ (função) \\
      % proposição $P \wedge Q$ & tipo de produto $P \times Q$\\
      % prova de uma proposição $P$ & termo $t$ do tipo $P$  (ou seja, $t:P$)\\
      % proposição $P$ é provável & tipo $P$ é habitado por algum termo \\
      % \hline
   % \end{tabular}
   % \label{tab:curry}
% \end{table}