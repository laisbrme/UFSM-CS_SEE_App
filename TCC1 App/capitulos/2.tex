\chapter{FUNDAMENTAÇÃO TEÓRICA}

Este capítulo apresenta uma breve revisão da literatura a respeito das ferramentas didáticas já existentes, sendo utilizadas na didática de Subestações de Energia Elétrica. % Além de descrever uma subestação e ver quais seus barramentos e suas manobras comuns.

\section{Ferramentas Computacionais}

Segundo Córdova Júnior \citeyearpar{cordova2018fundamentos}, os softwares são categorizados em dois grandes grupos: os softwares básicos e os softwares aplicativos. Os softwares básicos são programas que gerenciam todo o funcionamento do computador, além de fornecer uma interface com o usuário. Os softwares aplicativos são programas com funções específicas, que nos auxiliam a desenvolver alguma tarefa, como editar um texto ou realizar um cálculo.

Segundo Tajra \citeyearpar{tajra2012informatica}, a utilização de um software está diretamente relacionada à capacidade de percepção do professor em relacionar a tecnologia à sua proposta educacional. Por meio dos softwares podemos ensinar, aprender, simular, estimular a curiosidade ou, simplesmente, produzir trabalhos com qualidade.

As ferramentas pedagógicas, de forma geral, servem para facilitar o processo de aprendizagem, e esse termo depende da intenção e da finalidade de quem o utiliza, e contribuir com a educação efetiva do aluno. Dito isto, serão apresentadas algumas ferramentas 


\subsection{Ferramenta Didática de Subestações Elétricas SEUL}

Diante da importância do aprendizado, entendimento e da necessidade de
executar um projeto de uma subestação foi criada a \textit{Ferramenta Didática de Subestações Elétricas, SUEL} que tem como objetivo agregar, de forma prática e fácil, as informações relevantes quanto aos diferentes tipos de arranjos dessas plantas \cite{holanda2016suel}.

Para o desenvolvimento da ferramenta didática foi utilizado o programa \textit{Adobe Animate CC 2015} desenvolvido pela empresa norte-americana \textit{Adobe Systems Incorporated} \cite{holanda2016suel}.

Segundo Holanda \citeyearpar{holanda2016suel} o \textit{Adobe Animate CC} é uma plataforma de desenvolvimento baseado em linha de tempo, onde é possível criar animações vetoriais, conteúdo multimídia, aplicativos e jogos; possui um ambiente gráfico com ferramentas de desenho e ilustração, e um ambiente de programação, que permite adicionar interatividade e manipulação de dados ao conteúdo desenvolvido.


\subsection{FEUPowerTool: ferramanta pedagógica para manobras em subestações}

Segundo Ramos \citeyearpar{ramos2010feupowertool}, a criação de uma aplicação didática para a simulação de manobras de subestações [...] nasce pelo fato das tecnologias de apoio e suporte à formação nesta área serem escassas e obsoletas.

Existiu a necessidade de criar uma ferramenta que permitisse maior liberdade tanto para o utilizador como para o programador \cite{ramos2010feupowertool}. Sendo que esta "liberdade" só se consegue com um Ambiente de Desenvolvimento Integrado (IDE - Integrated Development Environment). Assim é possível desenvolver um ambiente gráfico que possibilita a criação de qualquer circuito, com a noção de seccionador, que distingue um seccionador de um disjuntor e detecta as manobras efetuadas durante a simulação \cite{ramos2010feupowertool}.

Segundo Lazarus and Free Pascal Team \cite{lazarus}, Lazarus é um IDE compatível com multiplataforma Delphi para Free Pascal. Free Pascal é um compilador de Licença Pública Geral (GLP - General Public License) projetado para ser capaz de entender e compilar a sintaxe Delphi, que é uma Programação Orientada a Objetos (OOP - Object-Oriented Programming).Foi este o programa escolhido, para o desenvolvimento da aplicação \cite{ramos2010feupowertool}.


\subsection{Simulador de uma subestação elétrica para ensino de princípios básicos de eletricidade}

O trabalho de Silva \citeyearpar{silva2017simulador} propões o desenvolvimento de um simulador com base no sistema de transmissão de energia da Eletrobrás/Eletronorte, onde o aluno poderá através de uma tela simulada do SAGE (Sistema Aberto de Gerenciamento de Energia) fazr operações com os disjuntores, com abertura e fechamento de carga, integrando-o a uma plataforma Arduino, onde este irá interpretar os comandos da parte do \textit{software} do simulador e convertê-los em sinais analógicos que acionarão o LED que representará a passagem de carga, em caso de ativado, ou não em caso de desativado para um determinado centro urbano.

A ferramenta utilizada para o desenvolvimento do projeto foi o ambiente 2D da plataforma Unity , que segundo Silva \citeyearpar{silva2017simulador} permite a criação de jogos e simuladores em 2D, apresentando uma interface muito simples e amigável, tendo como objetivo permitir a facilidade no desenvolvimento de jogos ou simuladores de diversos tipos e ainda outros sistema de visualização.


\subsection{Aplicação Informática para Dimensionamento de Barramentos em Subestações}

O trabalho de Tavares \citeyearpar{tavares2015aplicaccao} propõem a construção de uma aplicação informática, na qual efetuará automaticamente os cálculos necessários à validação dos barramentos escolhido pelo utilizador.

O \textit{software} a ser utilizado para construção da aplicação de dimensionamento dos barramentos foi o \textit{Microsoft Access} \cite{tavares2015aplicaccao}. Segundo Tavares \citeyearpar{tavares2015aplicaccao}, esta escolha deveu-se ao fato de as características do \textit{software} irem exatamente ao encontro dos objetivos propostos para a aplicação, pois trata-se de um \textit{software} de criação e gestão de base de dados, amplamente utilizado no mercado, de fácil interação com o utilizador.

A configuração de formulários e programas de ações será realizada em linguagem \textit{Visual Basic for Applications} (VBA) e \textit{Structured Query Language} (SQL) \cite{tavares2015aplicaccao}.


\subsection{AUTOMAÇÃO DE MANOBRAS EM SUBESTAÇÕES DE
TRANSMISSÃO DE ENERGIA ELÉTRICA}

O trabalho de Dias \citeyearpar{dias2017automaccao} propõem uma estratégia baseada na automação de etapas da tarefa para minimizar o erro durante a realização de manobras em um SEP, concretizada a partir do desenvolvimento de uma ferramenta de software. 

Para o desenvolvimento das funcionalidades da interface foi utilizada a linguagem de programação de alto nível Python [www.python.org e Barry (2011)] que se fundamenta na abordagem orientada a objetos, sendo esta escolha motivada pela simplicidade dos códigos e por ser uma linguagem nativa do sistema operacional Linux, este utilizado como sistema de suporte ao supervisório SAGE \cite{dias2017automaccao}.


%\subsection{IVERSON SOZO}

%Referindo-se ao dimensionamento de uma malha de aterramento, este trata-se de um método iterativo de cálculo, considerado repetitivo e demorado dependendo da situação de projeto \cite{sozo2014desenvolvimento}. Para resolver esta questão, são desenvolvidos softwares capazes de realizar as iterações em um tempo menor e mais preciso do que quando projetado “manualmente” \cite{sozo2014desenvolvimento}.

%o presente trabalho prevê a criação de uma ferramenta livre para cálculo de malha de aterramento na plataforma MATLAB (MATrix LABoratory) \cite{sozo2014desenvolvimento}.


\begin{table}[!hbt] % [htb]-> here, top, bottom
   \centering   % tabela centralizada
   \setlength{\arrayrulewidth}{1\arrayrulewidth}  % espessura da  linha
   \setlength{\belowcaptionskip}{5pt}  % espaço entre caption e tabela
   \caption{Tipo de licença das plataformas citadas}
   \begin{tabular}{l|l} % c=center, l=left, r=right 
      \hline
      \textbf{Plataforma} & \textbf{Licença} \\
      \hline
      Adobe Animate CC & Comercial  \\
      Lazarus & Código aberto \\
      Unity2D & Proprietário \\
      Microsoft Access & Comercial \\
      Python & Código Aberto \\
      \hline
   \end{tabular}
   \label{tab:plataformas}
\end{table}




%%=============================================================================
% \section{Subestações de Energia Elétrica}
% \subsection{Barramentos}
% \subsection{Manobras}

%%=============================================================================
% \section{Linguagem de Programação Python}
% \subsection{Interface Gráfica do Usuário}

%%=============================================================================